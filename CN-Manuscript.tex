%
%  =======================================================================
%  ····Y88b···d88P················888b·····d888·d8b·······················
%  ·····Y88b·d88P·················8888b···d8888·Y8P·······················
%  ······Y88o88P··················88888b·d88888···························
%  ·······Y888P··8888b···88888b···888Y88888P888·888·88888b·····d88b·······
%  ········888······"88b·888·"88b·888·Y888P·888·888·888·"88b·d88P"88b·····
%  ········888···d888888·888··888·888··Y8P··888·888·888··888·888··888·····
%  ········888··888··888·888··888·888···"···888·888·888··888·Y88b·888·····
%  ········888··"Y888888·888··888·888·······888·888·888··888··"Y88888·····
%  ·······························································888·····
%  ··························································Y8b·d88P·····
%  ···························································"Y88P"······
%  =======================================================================
% 
%  -----------------------------------------------------------------------
% Author       : 焱铭
% Date         : 2023-07-01 15:54:14 +0800
% LastEditTime : 2023-08-04 12:09:42 +0800
% FilePath     : \SCI-LaTeX-Submission-Process-for-Elsevier\CN-Manuscript.tex
% Description  : Manuscript-CN ⇒ Manuscript-CN ⇒ Major-Revision ⇒ Minor-Revision ⇒ Accepted Manuscript ⇒ Final Manuscript
%  -----------------------------------------------------------------------
%
\special{dvipdfmx:config z 0}                                                                                % XeLaTeX取消PDF压缩,加快编译速度
% \pdfcompresslevel=0                                                                                          % PdfLaTeX取消PDF压缩,加快编译速度
% \pdfobjcompresslevel=0                                                                                       % LuaLaTeX取消PDF压缩,加快编译速度
\documentclass[preprint,5p,sort&compress,times,10pt]{elsarticle}                                             % preprint review  % compress压缩引用序号 
%% Use the option review to obtain double line spacing
%% \documentclass[authoryear,preprint,review,12pt]{elsarticle}

%% Use the options 1p,twocolumn; 3p; 3p,twocolumn; 5p; or 5p,twocolumn
%% for a journal layout:
%% \documentclass[final,1p,times]{elsarticle}
%% \documentclass[final,1p,times,twocolumn]{elsarticle}
%% \documentclass[final,3p,times]{elsarticle}
%% \documentclass[final,3p,times,twocolumn]{elsarticle}
%% \documentclass[final,5p,times]{elsarticle}
%% \documentclass[final,5p,times,twocolumn]{elsarticle}

% ------------------------------------------------------------------------------------------------------------
%                                                  前言设置
% ------------------------------------------------------------------------------------------------------------
% 公式
\usepackage{amsmath}                                                                                         % 公式输入需要 equation 以及多行公式 [fleqn] 所有公式左对齐
\usepackage{amssymb}                                                                                         % The amsthm package provides extended theorem environments
\usepackage{xcolor}                                                                                          % 修稿标注时所需颜色宏
\usepackage{lineno}                                                                                          % 该宏包可显示行号
\usepackage{microtype}                                                                                       % 用于微调字距、字间距和断行,对英文、中文等文本进行优化,使得断行更加美观和平衡

% -----------------------------------------------------
%                        中文高亮文本设置               *******************************************************
% -----------------------------------------------------
\usepackage{ctex}                                                                                            % 中文宏包,取消注释即可输入中文内容 
\usepackage{ifluatex}                                           
\ifluatex                                                                                                    % 设置使用 lualatex 引擎时高亮颜色
    \usepackage{emoji}                                          
    \usepackage{luacolor,lua-ul}                                                                             % lualatex需要使用lua-ul宏包进行高亮设置
    \newcommand{\gl}[1]{\highLight{#1}}                                         
\else                                                                                                        % 设置使用 xelatex 引擎时高亮颜色
    \usepackage{xeCJKfntef}                                                                                  % 中文高亮设置,导入此包支持汉字特殊下划线效果 
    \newcommand{\gl}{\CJKsout*[thickness=2.5ex,format=\color{yellow}]}                                       % 新定义高亮命令\gl 用来高亮显示中文
\fi
% -----------------------------------------------------
%                        表格格式设置
% -----------------------------------------------------
\usepackage[caption=false,farskip=0pt,labelfont={bf}]{subfig}                                                % 设置子图所需宏包
\usepackage{booktabs}                                                                                        % 三线表
\usepackage{array}                                                                                           % 设置表格单元宽度
\usepackage{multirow}                                                                                        % 表格行合并单元格设置
\usepackage{threeparttable}                                                                                  % 添加表格注释
                                        
\usepackage[labelfont={bf}]{caption}                                                                         % 题注字体加粗
\usepackage{setspace}                                                                                        % 使用间距宏包

% -----------------------------------------------------
%                        术语表设置
% -----------------------------------------------------
\usepackage{ifthen}
\usepackage{framed}                                                                                          % 导入 framed 宏包,用于给内容添加框架效果
\usepackage{multicol}                                                                                        % 导入 multicol 宏包,提供多栏环境
\usepackage[]{nomencl}                                                                                       % 导入 nomencl 宏包,用于生成符号说明
\usepackage{etoolbox}                                                                                        % 导入 etoolbox 宏包,提供处理 TeX 和 LaTeX 代码的工

% 术语表详细设置
\renewcommand*\nompreamble{\begin{multicols}{2}}                                                             % 设置术语表为双栏显示
\renewcommand*\nompostamble{\end{multicols}}

% 更改术语之间的垂直行间距
\newlength{\nomitemorigsep}
\setlength{\nomitemorigsep}{\nomitemsep}
\setlength{\nomitemsep}{-1.25\parsep} % Baseline skip between items

% 创建术语表分组
\renewcommand{\nomgroup}[1]{%
\ifthenelse{\equal{#1}{A}}{\vspace{6pt} \item[\textbf{Greek symbols}]}{%
\ifthenelse{\equal{#1}{B}}{\vspace{6pt} \item[\textbf{Subscripts}]}{%
\ifthenelse{\equal{#1}{C}}{\vspace{6pt} \item[\textbf{Abbreviations}]}{}
}}

\itemsep\nomitemsep                                                                                          % 应用上面设置的术语垂直行间距
}
\makenomenclature                                                                                            % 打印术语表

% -----------------------------------------------------
%                        链接设置
% -----------------------------------------------------
\usepackage{hyperref}                                                                                        % 对目录生成链接,注:该宏包可能与其他宏包冲突,故放在所有引用的宏包之后
\hypersetup{                                                                                                 % 将链接文字带颜色
    colorlinks = true,
    citecolor=blue, 
    linkcolor=black, 
    urlcolor=black,
	bookmarksopen = true,                                                                                    % 展开书签
	bookmarksnumbered = true,                                                                                % 书签带章节编号
	pdftitle = title,                                                                                        % 标题
	pdfauthor = YanMing                                                                                      % 作者
    }
    %\setlength{\mathindent}{0pt} % 将公式的缩进调整为0

% -----------------------------------------------------
%                        题注及引用设置
% -----------------------------------------------------
\usepackage[capitalise,nameinlink]{cleveref}                                                                 % 引用宏包 ,该宏包要在 hyperref后引用   
                                                                                                             % capitalise 将引用的单词首字母大写
                                                                                                             % nameinlink 将引用的名称包含在超链接中
                                                                                                             % noabbrev 使用完整的引用名称,而不使用缩写

% 关于题注的设置
\captionsetup[subfigure]{subrefformat=simple,labelformat=simple,listofformat=subsimple}
\renewcommand\thesubfigure{(\alph{subfigure})}                                                               % 该代码与上一行 设置引用前缀子图序号带圆括号
\renewcommand{\figurename}{Fig.}                                                                             % 题注中使默认的图名字改为图显示
\renewcommand{\tablename}{Table}                                                                             % 题注中使默认的表名字改为表显示





% -----------------------------------------------------
%                        自定义命令                     *******************************************************
% -----------------------------------------------------
\newcommand{\Process}{1-Manuscript-CN}
\graphicspath{{\Process/Pictures/},}                                                                         % 图片所在文件夹,可放置多个文件夹,用,分隔。
\newcommand{\Nomenclature}{\input{\Process/Section/Nomenclature}}                                            % 定义新命令\Nomenclature用于包含术语表的内容
\newcommand{\eqtref}[1]{Eq.\ref{#1}}                                                                         %定义新命令:公式引用.引用之前需要在公式栏添加label,引用格式为\eqtref{label}
% -----------------------------------------------------
%                        目标期刊
% -----------------------------------------------------
\journal{Applied Thermal Engineering}                                                                        % 输入要投稿的期刊
% ------------------------------------------------------------------------------------------------------------
%                                                  正文内容
% ------------------------------------------------------------------------------------------------------------
\begin{document}
% \linenumbers                                                                                               % 增加行号 提交修改稿时用

% -----------------------------------------------------
%                        封面内容
% -----------------------------------------------------
\begin{frontmatter}


    \title{Paper Title}                                                                                      % 文章标题

    \author[rvt]{Yan Ming\corref{cor2}}
    \author[rvt]{Yan Ming\corref{cor2}}
    \author[rvt]{Yan Ming\corref{cor1}}%\fnref{fn1}
    \ead{your-email@guet.edu.cn}
    \author[rvt]{Yan Ming}%\fnref{fn1}


    \cortext [cor1]{Corresponding author}                                                                     % 通讯作者角标
    \cortext [cor2]{These authors contributed to the work equllly and should be regarded as co-first authors.}% 共同一作声明,没有可去掉
    \fntext[fn1]{This is the specimen author footnote.}                                                       % 关于作者的介绍之类,没有可去掉
    \affiliation[rvt]{organization={Guilin University Of Electronic Technology},                              % 所在学院和学校
        addressline={No.1 Jinji Road, Qixing District},
        city={Guilin},
        postcode={541004},
        state={the Guangxi Zhuang Autonomous Region},
        country={China}}

    \begin{abstract}                                                                                          % abstract的内容不要为空,否则参考文献引用不会显示
        A new microchannel heat sink with embedded modules with ribs and pin-fins is proposed as an effective solution to realize microchannel heat dissipation within low-temperature co-fired ceramic substrates.
    \end{abstract}

    % 图片摘要
    \begin{graphicalabstract}
        \centering
        \includegraphics*[width=1 \textwidth]{Schematic.jpg}
    \end{graphicalabstract}


    \begin{highlights}
        \item Microchannel cooling in a Low-Temperature Co-fired Ceramic substrate;
        \item A comparison with the performance of similar designs;
        \item Numerical study of three parameters of this microchannel heat sink.
    \end{highlights}

    \begin{keyword}
        Microchannel heat sink \sep Embedded modules \sep LTCC \sep Rib \sep Pin-fin
    \end{keyword}

\end{frontmatter}


\input{\Process/Section/Section1.tex}

\input{\Process/Section/Section2.tex}

\input{\Process/Section/Section3.tex}

%
%  =======================================================================
%  ····Y88b···d88P················888b·····d888·d8b·······················
%  ·····Y88b·d88P·················8888b···d8888·Y8P·······················
%  ······Y88o88P··················88888b·d88888···························
%  ·······Y888P··8888b···88888b···888Y88888P888·888·88888b·····d88b·······
%  ········888······"88b·888·"88b·888·Y888P·888·888·888·"88b·d88P"88b·····
%  ········888···d888888·888··888·888··Y8P··888·888·888··888·888··888·····
%  ········888··888··888·888··888·888···"···888·888·888··888·Y88b·888·····
%  ········888··"Y888888·888··888·888·······888·888·888··888··"Y88888·····
%  ·······························································888·····
%  ··························································Y8b·d88P·····
%  ···························································"Y88P"······
%  =======================================================================
% 
%  -----------------------------------------------------------------------
% Author       : 焱铭
% Date         : 2023-07-04 21:20:55 +0800
% LastEditTime : 2023-07-05 13:40:59 +0800
% Github       : https://github.com/YanMing-lxb/
% FilePath     : \SCI-LaTeX-Submission-Process-for-Elsevier\2-Manuscript-EN\Section\Section4.tex
% Description  : 
%  -----------------------------------------------------------------------
%


\section{Results and discussion}

\subsection{Numerical validations}
To verify the accuracy of this simulation scheme, the numerical results are compared with the experimental data of several experiments \cite{Zhang.Wu.ea_2022,Qu.Mudawar_2002,Yang.Wang.ea_2017}, as shown in \cref{fig:Verification}.
\begin{figure*}[htbp]
    \centering
    \scriptsize% 设置字体大小
    \subfloat{
        \label{fig:Zhang}
        \includegraphics[width=0.35 \textwidth]{V-Zhang-T.pdf}} % 两个\subfloat之间加回车,图片会换行\hspace{2mm}
    \subfloat{
        \label{fig:Qu}
        \includegraphics[width=0.4 \textwidth]{V-Qu-PT.pdf}}

    \subfloat{
        \label{fig:Yang}
        \includegraphics[width=0.75 \textwidth]{V-Yang-T.pdf}}
    \caption{Numerical validations.
        (a)Zhang et al. \cite{Zhang.Wu.ea_2022} of the microchannel heat sink at different mass flow rates for the variation of the bottom surface temperature of the microchannel heat sink.
        (b)Variation of inlet and outlet temperature difference and pressure drop in microchannels of Qu and Mudawar \cite{Qu.Mudawar_2002} at different Reynolds numbers.
        (c)Yang et al. \cite{Yang.Wang.ea_2017} of the pin-fin heat sink with the highest temperature on the bottom surface of the pin-fin heat sink for different pin-fin shapes.}
    \label{fig:Verification}
\end{figure*}

\subsection{Effect of geometric prameters on hydrothermal performance}

In order to explore the influence of the geometric




\subsubsection{The effect of relative rib height}

\subsubsection{Performance analysis}



\begin{table*}[!ht]
    \renewcommand{\arraystretch}{1.5} % 调整行距
    \centering
    \scriptsize
    \caption{Comparison with other solutions}
    \begin{threeparttable}
        \begin{tabular}{m{1.8cm}<{\raggedright}m{2.5cm}<{\raggedright}m{1.5cm}<{\raggedright}m{1.5cm}<{\raggedright}m{1cm}<{\raggedright}m{1.5cm}<{\raggedright}m{1.7cm}<{\raggedright}} \toprule
            Reference & Cooling methods & Heating area ($mm^2$) & Heat power ($W$) & Flow rate ($ml/min$) & Inlet pressure ($KPa$) & Maximum temperature ($^\circ C$) \\ \midrule
            \multirow{2}{*}{Zhang et al. \cite{Zhang.Zhang.ea_2015}} & \multirow{2}{2cm}{parallel cooling microchannels} & \multirow{2}{*}{$22\times 22$} & \multirow{2}{*}{75} & \multirow{2}{*}{58.1} & 330 & 99.52 \\ \cline{6-7}
            & & & & & $0.096^*$ & $55.7^*$ \\[5pt]
            \multirow{2}{*}{Yin et al. \cite{Yin.Li.ea_2019}} & \multirow{2}{2.2cm}{LTCC with embedded metal pillar arrays} & \multirow{2}{*}{$21\times 21$} & \multirow{2}{*}{20} & \multirow{2}{*}{18.85} & 7.12 & 74.85 \\ \cline{6-7}
            & & & & & $0.021^*$ & $57.35^*$ \\[5pt]
            \multirow{2}{*}{Liu et al. \cite{Liu.Jin.ea_2016}} & \multirow{2}{2.2cm}{LTCC with via holes and liquid metal} & \multirow{2}{*}{$10\times 10$} & \multirow{2}{*}{$30^{**}$} & \multirow{2}{*}{70} & \quad - & 83.85 \\ \cline{6-7}                                                                  
            & & & & & $0.138^*$ & $95.74^*$ \\[5pt]
            \multirow{2}{*}{Yu et al. \cite{YU.HAN.ea_2018}} & \multirow{2}{2.2cm}{LTCC with dual-layer spirals microchannels} & \multirow{2}{*}{$2\times 10$} & \multirow{2}{*}{$23^{**}$} & \multirow{2}{*}{45} & 370.7 & 84.85 \\ \cline{6-7}
            & & & & & $0.071^*$ & $55.54^*$\\\bottomrule
        \end{tabular}
        *\,\, The heat sink is MCHS-RPFEM and the coolant is deionized water.\\
        ** Heat flux($W/cm^2$)
    \end{threeparttable}
    \label{tab:Literature-comparison}
\end{table*}



%在三线表中添加分栏线
%\begin{table}[htbp]
%    \centering
%    \scriptsize
%    \caption{natural frequencies from the infinite element analysis}
%    \begin{tabular}{ccccccccc}
%        \toprule
%     
%         &  & \multirow{2}{*}{Model} &  &  &  &   & \multicolumn{2}{c}%{Frequency(Hz)}   \\ 
%         \cline{8-9}
%        
%         &  &    &  &  &  &  & \multirow{2}{*}{Model} & \multirow{2}{*}%{Model\&Thermal} \\ 
%         \midrule
%         &  &  1 &  &  &  &  & 2954.2 & 2576.7 \\ 
%         &  &  2 &  &  &  &  & 4808.8 & 4313.1 \\ 
%         &  &  3 &  &  &  &  & 4834.9 & 4323.2 \\ 
%         &  &  4 &  &  &  &  & 10921  & 10626  \\ 
%         &  &  5 &  &  &  &  & 11171  & 10837  \\ 
%         &  &  6 &  &  &  &  & 13402  & 13137  \\ 
%        \bottomrule
%        \label[table]{temp_profile_info}
%    \end{tabular}
%\end{table}
%
%




\input{\Process/Section/Section5.tex}



% section 加* 表示不显示序号
\section*{Declaration of Competing Interest}
The authors declare that they have no known competing financial interests or personal relationships that could have appeared to influence the work reported in this paper.
\section*{Formatting of funding sources}
This research did not receive any specific grant from funding agencies in the public, commercial, or not-for-profit sectors.

%% The Appendices part is started with the command \appendix;
%% appendix sections are then done as normal sections
%% \appendix

%% 参考文献
\bibliographystyle{elsarticle-num}                                                                          % 加载参考文献样式文件
\bibliography{References}                                                                                   % 参考文献bib库

\end{document}
\endinput
